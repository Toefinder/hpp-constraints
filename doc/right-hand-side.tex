\documentclass {article}

\newcommand\conf{\mathbf{q}}
\newcommand\fr{\mathcal{F}}
\newcommand\reals{\mathbf{R}}

\title {Explicit and implicit relative pose constraints}
\author {Florent Lamiraux}
\date {}

\begin {document}
\maketitle

\section {Introduction}

We wish to represent relative pose constraints between two given frames $\fr_1$ and $\fr_2$ with the following properties:
\begin{enumerate}
\item setting Boolean masks should enable the user to define constraints with
  symmetries : only orientation is constrained, only the position of the origin of $\fr_2$ in $\fr_1$ is constrained, rotation around an axis is free ...
\item when the relative pose is fully constrained, and $\fr_2$ is hold by a free-floating joint, it should be possible to express the position of this joint with respect to the position of $\fr_1$.
\end{enumerate}

\section {Explicit representation}

To satisfy item 2 above, we define an explicit constraint as follows. We use the following notation.
\begin {itemize}
\item $J_1\in SE(3)$ the position of Joint 1 in global frame,
\item $J_2\in SE(3)$ the position of Joint 2 in global frame,
\item $F_{1/J_1}$ the constant position of $\fr_1$ with respect to Joint 1,
\item $F_{2/J_2}$ the constant position of $\fr_2$ with respect to Joint 2,
\item $\conf_{out}$ the vector representation of $J_2$.
\end {itemize}
When the constraint is satisfied, we have
\begin {eqnarray*}
  J_1 F_{1/J_1} &=& J_2 F_{2/J_2} \\
  J_2 &=& J_1 F_{1/J_1} F_{2/J_2}^{-1}
\end {eqnarray*}
This gives rise to the following explicit constraint:
\begin {eqnarray*}
  \conf_{out} &=& {J_1} (\conf_{in}) F_{1/J_1} F_{2/J_2}^{-1}
\end {eqnarray*}

The right hand side of this constraint is
\begin {eqnarray}
  rhs_{expl} &=& \conf_{out} - ({J_1} (\conf_{in})F_{1/J_1} F_{2/J_2}^{-1})\\
  \label{explicit-rhs}
  &=& \log_{SE(3)} \left(F_{2/J_2} F_{1/J_1}^{-1} {J_1} (\conf_{in})^{-1} \conf_{out}\right)
\end {eqnarray}

\section {Implicit Representation}

To satisfy item 1 in the introduction, we define the following implicit constraint:
\begin {eqnarray}
  rhs_{impl} &=& \log_{\reals^3\times SO(3)} \left((J_1 F_{1/J_1})^{-1}J_2 F_{2/J_2}\right)\\
  &=&\log_{\reals^3\times SO(3)} \left((J_1 F_{1/J_1})^{-1}\conf_{out} F_{2/J_2}\right)
\end {eqnarray}

\section {Conversion between right hand sides}

\begin {eqnarray*}
rhs_{expl} &=& \log_{SE(3)} \left(F_{2/J_2} F_{1/J_1}^{-1} J_1^{-1} J_2\right)\\
rhs_{impl} &=& \log_{\reals^3\times SO(3)} \left(F_{1/J_1}^{-1} J_1^{-1}J_2 F_{2/J_2}\right)\\
%\exp_{SE(3)} rhs_{expl} &=& F_{2/J_2} F_{1/J_1}^{-1} J_1^{-1} J_2\\
\exp_{\reals^3\times SO(3)} (rhs_{impl}) &=&  F_{1/J_1}^{-1} J_1^{-1}J_2 F_{2/J_2} \\
F_{2/J_2}\exp_{\reals^3\times SO(3)} (rhs_{impl})  F_{2/J_2}^{-1} &=&  F_{2/J_2} F_{1/J_1}^{-1} J_1^{-1}J_2\\
rhs_{expl} &=& \log_{SE(3)}\left(F_{2/J_2}\exp_{\reals^3\times SO(3)} (rhs_{impl})  F_{2/J_2}^{-1}\right)
\end {eqnarray*}
\end {document}

